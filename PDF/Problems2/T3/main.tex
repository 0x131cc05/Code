\documentclass[12pt]{article}
\usepackage{xeCJK}
\usepackage{float}
\usepackage{indentfirst}
\usepackage{booktabs}
\usepackage{geometry}  
\usepackage{titlesec}   
\usepackage{array}
\usepackage{multirow}
\usepackage{listings}
\usepackage{titlesec} 
% \pagestyle{fancy} 
\geometry{left=2.5cm,right=3.5cm,top=2.5cm,bottom=2.5cm}

\setlength{\parindent}{2em}
\linespread{1.3}

\date{}
\begin{document}
\newpagestyle{main}{            
    \sethead{膜您赛}{}{\sectiontitle}     %设置页眉
    \setfoot{}{第\thepage 页,共2页}{}      %设置页脚,可以在页脚添加 \thepage  显示页数
    \headrule                                      % 添加页眉的下划线
    \footrule                                       %添加页脚的下划线
}
\pagestyle{main}

\setcounter{section}{2}

\section{T3 风暴}

当希望凋零,有人仍选择囿于黑暗。\par

一场规模史无前例的暴风雨在尼文境内肆虐,摧毁了本就为数不多的连接各个城市的道路。尼文的的城市可以抽象为一个$n\times m$的矩阵,每个坐标$(i, j)$处都有一座城市。一开始两个城市之间有一条\textbf{双向道路}连接当且仅当这两个城市的位置相邻,即这两个城市的坐标满足$|x_1 - x_2| + |y_1 - y_2| = 1$。\par
暴风雨停息之后,有若干条连接这些城市的道路被摧毁了。具体来说,每条道路都有独立的$p$的概率被摧毁,被摧毁的道路无法通行。\par
尼文的首都是坐标为$(1, 1)$的城市,而精灵之树则位于$(n, m)$。为了估计暴风雨给尼文带来的损失,你迫切地想知道,这两座城市不再连通的概率有多大。\par

\subsection{输入格式}

输入包含多组数据。\par
第一行一个整数$T$,表示数据组数。\par
接下来$T$行,每行两个整数$n, m$以及一个不超过$1$的小数$p$,表示矩阵的大小以及每条路径断开的概率。

\subsection{输出格式}

对于每组数据输出一行一个小数,表示$(1, 1)$与$(n, m)$不再连通的概率。\par
你的答案与标准答案的绝对误差不能超过$10^{-6}$。

\subsection{样例 1 输入}

\begin{lstlisting}
    3
    2 2 0.5
    3 3 0.4
    4 5 0.7
\end{lstlisting}

\subsection{样例 1 输出}

\begin{lstlisting}
    0.4375000000000
    0.4767412101120
    0.0071313208832
\end{lstlisting}

\subsection{提示}

$0.4375 = \frac{7}{16}$

\subsection{数据范围}

对于$10\%$的数据,保证$n, m = 2$。\par
对于$30\%$的数据,保证$n, m \leq 4, T = 1$。\par
对于$60\%$的数据,保证$m \leq 50$。\par
对于$100\%$的数据,$1\leq n\leq 8, 1\leq m \leq 10^{18}, T\leq 50, 0.1\leq p\leq 1$,且$p$最多为四位小数。\par

\end{document}