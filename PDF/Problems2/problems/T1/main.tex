\documentclass[12pt]{article}
\usepackage{xeCJK}
\usepackage{float}
\usepackage{indentfirst}
\usepackage{booktabs}
\usepackage{geometry}  
\usepackage{titlesec}   
\usepackage{array}
\usepackage{multirow}
\usepackage{listings}
\usepackage{titlesec} 
% \pagestyle{fancy} 
\geometry{left=2.5cm,right=3.5cm,top=2.5cm,bottom=2.5cm}

\setlength{\parindent}{2em}
\linespread{1.3}

\date{}
\begin{document}
\newpagestyle{main}{            
    \sethead{膜您赛}{}{\sectiontitle}     %设置页眉
    \setfoot{}{第\thepage 页,共2页}{}      %设置页脚,可以在页脚添加 \thepage  显示页数
    \headrule                                      % 添加页眉的下划线
    \footrule                                       %添加页脚的下划线
}
\pagestyle{main}

\section{T1 废墟}

光芒破碎,我们只剩下缕缕萤火。\par

古老的废墟中散落着大大小小的瓷器碎片,远古遗迹的神秘面纱也在一步一步揭开。在废墟中,你一共发现了$n$块碎片,每块碎片的大小都不一样,这$n$块碎片的大小构成了一个$1$到$n$的排列。\par
为了解开前人们留下的谜题,你将所有碎片都排列在了一起。前人们留下的唯一提示是,在最终的排列中,相邻两块碎片的大小之和不能超过$m$。但仅凭这一点提示根本不足以解开复杂的谜题,因此你决定尝试所有的排列方式。为了估计需要耗费的时间,你需要求出一共有多少种满足条件的排列方式。\par

\subsection{输入格式}

输入仅包含一行两个整数:$n, m$。

\subsection{输出格式}

输出一个数,表示合法排列的数量对$998244353$取模之后的值。

\subsection{样例 1 输入}

\begin{lstlisting}
    4 5
\end{lstlisting}

\subsection{样例 1 输出}

\begin{lstlisting}
    4
\end{lstlisting}

\subsection{提示}

合法的排列方式有如下四种:$[2, 3, 1, 4], [3, 2, 1, 4], [4, 1, 2, 3], [4, 1, 3, 2]$。

\subsection{样例 2 输入}

\begin{lstlisting}
    5 8
\end{lstlisting}

\subsection{样例 2 输出}

\begin{lstlisting}
    72
\end{lstlisting}

\subsection{数据范围}

对于$20\%$的数据,保证$n\leq 10$。\par
对于$40\%$的数据,保证$n\leq 20$。\par
对于$100\%$的数据,$1\leq n\leq 10^6, n < m < 2n$。

\end{document}