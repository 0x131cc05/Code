\documentclass[12pt]{article}
\usepackage{xeCJK}
\usepackage{float}
\usepackage{indentfirst}
\usepackage{booktabs}
\usepackage{geometry}  
\usepackage{titlesec}   
\usepackage{array}
\usepackage{multirow}
\usepackage{listings}
\usepackage{titlesec} 
% \pagestyle{fancy} 
\geometry{left=2.5cm,right=3.5cm,top=2.5cm,bottom=2.5cm}

\setlength{\parindent}{2em}
\linespread{1.3}

\date{}
\begin{document}
\newpagestyle{main}{            
    \sethead{膜您赛}{}{\sectiontitle}     %设置页眉
    \setfoot{}{第\thepage 页,共2页}{}      %设置页脚,可以在页脚添加 \thepage  显示页数
    \headrule                                      % 添加页眉的下划线
    \footrule                                       %添加页脚的下划线
}
\pagestyle{main}

\setcounter{section}{1}

\section{T2 魔法}

破镜终有重圆时。\par
精灵之树上住着$n$个小精灵,每个小精灵都有其唯一的一个$1$到$n$的编号。树上共有$n - 1$根树枝,每根树枝都连接着两个小精灵。你可以认为所有小精灵连成了一个树型的结构。\par
作为精灵之树的主人,你将施展$n$次魔法以抵御接下来的暴风雨。在第$i$次施法中,如果树上编号为$1, 2, \cdots , i$的小精灵恰好连成了一条简单路径,才能保证魔力的最大化,此时第$i$次施法是成功的。我们将施法成功的总次数称为这棵树的\textbf{魔力值}。\par
魔法总是在更新换代,为了保持精灵之树魔力的平衡,需要对精灵之树进行$q$次调整,每次调整会选择两个编号为$u, v$的\textbf{节点},然后交换住在这两个节点上的小精灵。\par
前任精灵之主的低语时时刻刻都在提醒着你时间紧迫,对于每一次调整你都需要尽快回答整棵树调整之后的魔力值是多少。\par

\subsection{输入格式}

第一行为一个整数$n$,表示精灵之树的大小。\par
第二行包含$n$个整数,第$i$个整数$p_i$表示一开始住在节点$i$的小精灵的编号。\par
接下来$n - 1$行,每行两个整数$a_i, b_i$表示树上的一条边。\par
接下来为一个整数$q$,表示调整的次数。\par
接下来$q$行,每行两个整数$u, v$,表示一次调整。

\subsection{输出格式}

输出$q$行,第$i$行为第$i$次调整之后整棵树的魔力值。

\subsection{样例 1 输入}

\begin{lstlisting}
    4
    1 2 3 4
    1 2
    2 3
    2 4
    3
    1 4
    2 4
    1 2
\end{lstlisting}

\subsection{样例 1 输出}

\begin{lstlisting}
    3
    3
    1
\end{lstlisting}

\subsection{数据范围}

对于$20\%$的数据,满足$n, q\leq 100$。\par
对于$40\%$的数据,满足$n, q\leq 2000$。\par
对于另$20\%$的数据,$a_i = i, b_i = i + 1$。\par
对于$100\%$的数据,$2\leq n, q\leq 5\times 10^5$,保证$p$是一个$1$到$n$的排列。\par
\textbf{数据不保证$u, v$不相同}

\end{document}