\documentclass{beamer}
\usetheme{CambridgeUS}
\usecolortheme{beaver}

\usepackage{xeCJK}
\usefonttheme[onlymath]{serif}

\begin{document}

\begin{frame}
    \title{良心题}
    \author{An\_Account}
    \begin{titlepage}
    \end{titlepage}
\end{frame}

\begin{frame}
    \frametitle{礼物}
    \begin{itemize}
        \item 给定$n, m, k$,你需要求出$n$元手环的数量,满足其中恰有$m$个位置被选中,且不能连续选中$k$个位置。对$998244353$取模。
        \item 两个手环不同当且仅当它们不能通过旋转重合。
        \item $T\leq 5, n\leq 10^6, k\leq m\leq n$
    \end{itemize}
\end{frame}

\begin{frame}
    \frametitle{礼物}
    \begin{itemize}
        \item 考虑 Polya 。记$f(a, b)$表示长度为$a$的链,其中有$b$个位置被选中,且不能连续选中$k$个位置的方案数。(包括首尾)
        \item 容易得出答案为
        $$\begin{aligned}
            \sum_{d\mid n, m} f(\frac nd, \frac md) \varphi(d)
        \end{aligned}$$
        \item 现在我们需要快速计算$f$。
    \end{itemize}
\end{frame}

\begin{frame}
    \frametitle{礼物}
    \begin{itemize}
        \item 将$f(a, b)$看作先固定$a - b$个不选的珠子,将剩下的珠子插入间隙,使得每个间隙的珠子的数量都小于$k$的方案数。
        \item 固定$a - b, k$,可以得出$f$的生成函数 
        $$\begin{aligned}
            F(x) = (\sum_{i = 0}^{k - 1}x^i)^{a - b - 1} \times (\sum_{i = 0}^{k - 1}(i + 1)x^i)
        \end{aligned}$$
        \item 记$G(x) = \sum_{i = 0}^{k - 1} x^i$,则
        $$\begin{aligned}
            F(x) = G(x)^{a - b - 1}(G(x) + x^k)'
        \end{aligned}$$
    \end{itemize}
\end{frame}

\begin{frame}
    \frametitle{礼物}
    $$\begin{aligned}
        G(x) &= \frac{x^k - 1}{x - 1}\\
        G'(x) &= \frac{(kx^{k - 1} - 1)(x - 1) - (x^k - 1)}{(x - 1)^2}\\
        &= \frac{(k - 1)x^k - kx^{k - 1} + 1}{(x - 1)^2}\\
        F(x) &= \left(\frac{x^k - 1}{x - 1}\right)^{a - b - 1}\left(\frac{(k - 1)x^k - kx^{k - 1} + 1}{(x - 1)^2} + kx^{k - 1}\right)\\
        &= \frac{(x^k - 1)^{a - b - 1}}{(x - 1)^{a - b + 1}} \times (kx^{k + 1} - (k + 1)x^k + 1)
    \end{aligned}$$
\end{frame}

\begin{frame}
    \frametitle{礼物}
    \begin{itemize}
        \item 我们有$f(a, b) = [x^b]F(x)$,根据上面的式子可以发现我们只需要计算$\frac{(x^k - 1)^{a - b - 1}}{(x - 1)^{a - b + 1}}$某三项的值。
        \item 将分母分子同时二项式展开,可以发现分母只有$\frac bk$项可能有用。因此暴力计算一次$f(a, b)$的复杂度为$O(\frac bk)$。
        \item 复杂度$O(\frac{\sigma_1(\gcd(n, m))}{k}) \approx O(\frac{n\log\log n}{k})$
    \end{itemize}
\end{frame}

\begin{frame}
    \frametitle{Adi and the Matrix}
    \begin{itemize}
        \item 统计$n\times m$的$01$矩阵的数量,如果两个矩阵可以通过给行列重新分配编号重合则被视为相同。
        \item $n\times m\leq 550$
    \end{itemize}
\end{frame}

\begin{frame}
    \frametitle{Adi and the Matrix}
    \begin{itemize}
        \item 注意到题目数据范围比较特殊,$n\times m\leq 550$等价于$n, m$中较小的那个不会超过$23$。
        \item 对于$(i, j)$这个位置,考虑它在行列的置换。如果在行中它所在环的长度为$a$,在列中它所在的长度为$b$,那么它就需要$\operatorname{lcm}(a, b)$次轮换才能回到原位置,即每个不动点实际上都包含$\operatorname{lcm}(a, b)$个位置。
        \item 因此,考虑行内轮换中的一个环$A$以及列内轮换中的一个环$B$,显然它会涉及到$|A|\times|B|$个格子,而根据上面的讨论我们可以得出不动点的数量为$\frac{|A|\times |B|}{\operatorname{lcm}(|A|, |B|)} = \gcd(|A|, |B|)$。
    \end{itemize}
\end{frame}

\begin{frame}
    \frametitle{Adi and the Matrix}
    \begin{itemize}
        \item 显然如果组成两个轮换的每个环的环长相等,则两个轮换对答案的贡献相同。
        \item 不妨设$n < m$,则$n\leq 23$,我们可以爆搜$n$的划分方案。
        \item 假设在$n$的轮换中,长度为$a_i$的环有$b_i$个,那么置换的数量为
        $$\begin{aligned}
            \frac{n!}{\prod (a_i!)^{b_i}b_i!}\times \prod (a_i - 1)!^{b_i} = \frac{n!}{\prod a_i^{b_i}b_i!}
        \end{aligned}$$
    \end{itemize}
\end{frame}

\begin{frame}
    \frametitle{Adi and the Matrix}
    \begin{itemize}
        \item 暴力做法是,枚举行的拆分,枚举列的拆分,计算不动点数量之后再用刚刚那个式子算对应的置换个数。
        \item 事实上,我们可以只枚举行的拆分,将枚举列的拆分写成$dp$的形式,即$dp[i][j]$表示还剩$i$列可以拆分,每个环长至多为$j$的贡献之和。
        \item 将$2$的不动点个数次方以及对应的置换数量写进$dp$转移。
    \end{itemize}
\end{frame}

\begin{frame}
    \frametitle{经典题目}
    \begin{itemize}
        \item 完全图,每条边$k$染色,可以通过给顶点重新分配编号重合的图被视为相同的图,求方案数。
        \item $n\leq 60$
    \end{itemize}
\end{frame}

\begin{frame}
    \frametitle{经典题目}
    \begin{itemize}
        \item 仍然考虑枚举$n$的拆分,接下来我们将边分为两类:连接两个在同一个环上的点以及连接两个在不同环上的点。分别考虑这些边的不动点数量。
        \item 假设一个环长度为$k$,那么在所有只连接环内点的边中,本质不同的边只有$\lfloor\frac k2\rfloor$种。
        \item 对于两个环,假设第一个环长度为$a$,第二个环长度为$b$,可以发现一条边与$\operatorname{lcm}(a, b)$条边等价,因此不动点个数为$\gcd(a, b)$。
        \item 您会了。
    \end{itemize}
\end{frame}

\begin{frame}
    \frametitle{美术作业}
    \begin{itemize}
        \item 给出一棵基环\textbf{外向树},求出有多少种本质不同的给点$k$染色的方式,可以通过给顶点重新分配编号重合的染色方式被视为本质相同。
        \item $n\leq 10^5$
    \end{itemize}
\end{frame}

\begin{frame}
    \frametitle{美术作业}
    \begin{itemize}
        \item 本质上所有置换只有两种:交换子树中某个点的两个儿子以及转环。
        \item 先考虑如何计算一棵外向树的染色方式。
        \item 对于$u$,它的子树是无序的,对于$u$的每棵子树都求出染色方案以及hash,然后我们考虑hash相同的那些儿子。
        \item 假设这个相同的hash值为$h$,有$t$个儿子的hash等于$h$。设其中一个儿子子树内的染色方案数为$k$,那么我们可以将这些方案编号为$[1, k]$,每个子树从这些方案中选一个,统计排序后本质不同的选择方案的数量。
        \item 它相当于有$t$个完全相同的球放入$k$个不同的盒子的方案数,答案等于${k + t - 1\choose t}$。
    \end{itemize}
\end{frame}

\begin{frame}
    \frametitle{美术作业}
    \begin{itemize}
        \item 接下来考虑环的置换。
        \item 求出环上挂的每棵树的hash,对这个hash跑一遍KMP求最小循环节,那么每个置换旋转的距离都应是这个循环节的倍数。
        \item 然后就是普通polya。
    \end{itemize}
\end{frame}

\begin{frame}
    \frametitle{Geometric Progressions}
    \begin{itemize}
        \item 给定$n$个正整数对$(a_i, b_i)$,第$i$对数表示一个序列$\{a_i, a_ib_i, a_ib_i^2, \cdots \}$,问所有序列的最小公共元素,或者判断不存在。
        \item $n\leq 100, a_i, b_i\leq 10^9$
    \end{itemize}
\end{frame}

\begin{frame}
    \frametitle{Geometric Progressions}
    \begin{itemize}
        \item 对于每个$a_i, b_i$的质因子我们分开考虑,对于某个质因子$p$,公共元素所含的$p$的次数必然相等。
        \item 先考虑$n = 2$的情况,记$s(a, p)$表示$a$中$p$的次数,那么有
        $$\begin{aligned}
            s(a_i, p) + xs(b_i, p) = s(a_j, p) + y(b_j, p)
        \end{aligned}$$
        \item 对于每个质因子我们都可以列出一个这样的方程。如果线性无关的方程数量至少为$2$,那么我们就可以直接解出$x, y$,从而确定公共元素是啥。
    \end{itemize}
\end{frame}

\begin{frame}
    \frametitle{Geometric Progressions}
    \begin{itemize}
        \item 当$n > 2$时,依次合并前$i$个序列和第$i + 1$个序列。
        \item 如果合并过程中直接解出了$x, y$,那么求出这个公共元素,代入每个序列验证是否合法即可。
        \item 否则$p$这个质数的质数必然可以被表示为$kx + b$的形式,这里$b$是之前所有方程的最小非负整数解,$k$是一个类似于$\operatorname{lcm}$的东西。
    \end{itemize}
\end{frame}

\begin{frame}
    \frametitle{Geometric Progressions}
    \begin{itemize}
        \item 最后,关于二元二次方程有一个很有意思的结论。
        $$\begin{cases}a_1x + b_1y = c_1\\ a_2x + b_2y = c_2\end{cases} \Rightarrow x=\frac{\begin{vmatrix}c_1&b_1\\c_2&b_2\end{vmatrix}}{\begin{vmatrix}a_1&b_1\\a_2&b_2\end{vmatrix}},y=\frac{\begin{vmatrix}a_1&c_1\\a_2&c_2\end{vmatrix}}{\begin{vmatrix}a_1&b_1\\a_2&b_2\end{vmatrix}}$$
    \end{itemize}
\end{frame}

\begin{frame}
    \frametitle{Number of Binominal Coefficients}
    \begin{itemize}
        \item 统计有序对$(k, n)$的数量,满足$0\leq k\leq n\leq A$,且${n\choose k}$能被$p^a$整除。
        \item $1\leq p, a\leq 10^9, A\leq 10^{1000}$
    \end{itemize}
\end{frame}

\begin{frame}
    \frametitle{Number of Binominal Coefficients}
    \begin{itemize}
        \item 库默尔定理:组合数${n + m\choose m}$所含$p$的次数等于$n, m$在$p$进制下相加时的进位次数。
        \item 通过观察可以发现,$a$必然不会很大。具体来说,$a$不会超过$3500$。
        \item 记$dp[i][j][0/1][0/1]$表示从高到低位考虑,当前到了第$i$位,进位次数为$j$,当前$n + m$是否等于$A$,是否收到了$i + 1$位的进位。
        \item 转移写起来非常酸爽。
    \end{itemize}
\end{frame}

\begin{frame}
    \frametitle{简单数学题}
    \begin{itemize}
        \item 给定非负整数$x$,质数$p$,求最小的非负整数$n$,使得$f_n\equiv x\pmod p$。保证$p$是质数且个位为$1$或$9$。
        \item $T\leq 100, p\leq 2\times 10^9$
    \end{itemize}
\end{frame}

\begin{frame}
    \frametitle{简单数学题}
    \begin{itemize}
        \item 通过二次互反律可以得出,对于个位为$1$或$9$的质数,$5$必然是二次剩余。
        \item 首先我们知道
        $$\begin{aligned}
            f_n = \frac{1}{\sqrt 5}\left[\left(\frac{1 + \sqrt 5}{2}\right)^n - \left(\frac{1 - \sqrt 5}{2}\right)^n\right]
        \end{aligned}$$
        \item 用Cipolla解出$5$的平方根。
    \end{itemize}
\end{frame}

\begin{frame}
    \frametitle{简单数学题}
    \begin{itemize}
        \item 记$T = \frac{1 + \sqrt 5}{2}$,可以发现$\frac{1 - \sqrt 5}{2} = \frac {-1}T$。
        \item 接下来是一个关于$T^n$的二次方程。
        $$\begin{aligned}
            T^n - \frac{(-1)^n}{T^n} = f_n\sqrt 5
        \end{aligned}$$
        \item 讨论$n$的奇偶性,你可以假设$n$为奇数或偶数后分别跑一遍,那么问题是:如何判断这个方式是否存在一个$n$是偶数/奇数的解?
        \item 有两种方法,第一种是在BSGS时记录一下偶数的解以及奇数的解,另一种方法是直接解出$T^k\equiv 1\pmod p$的最小正整数解,判断$k$是否为奇数,可以发现如果$n$是原方程的解,那么$n + k$也是一个解。
    \end{itemize}
\end{frame}

\begin{frame}
    \frametitle{Perfect Square}
    \begin{itemize}
        \item 给定一个$n\times n$的矩阵,每个位置都有一些数,你需要选出一些数,使得每行每列都有奇数个数被选,且所选数的乘积是完全平方数。求方案数。
        \item $n\leq 20, a_{i, j}\leq 10^9$
    \end{itemize}
\end{frame}

\begin{frame}
    \frametitle{Perfect Square}
    \begin{itemize}
        \item 它们乘积的每个质因子的质数都是偶数,也就是说,我们可以将一个数用一个$01$序列表示,第$i$位表示$p_i$的次数的奇偶性。
        \item 把所有出现过的质因子拿出来,将每个数用$01$序列表示。如果忽略行列的限制,问题就等价于给你若干个数,问异或出$0$的方案数。可以用线性基很方便地求出。
        \item 现在考虑行列的限制,我们直接给每个序列加上$2n$个数,第$i$个变量表示行,第$i + n$个变量表示列,如果为$1$则表示在这一行/列上有奇数个数被选。
        \item 相当于询问异或出$000\cdots01111\cdots 1$的方案数。
    \end{itemize}
\end{frame}

\begin{frame}
    \frametitle{杜老师}
    \begin{itemize}
        \item 多组询问,每次给定$L, R$,求出有多少种在区间内选数的方式使得它们的乘积是完全平方数。
        \item $T\leq 100, R_i\leq 10^7, \sum R_i - L_i + 1\leq 6\times 10^7$
    \end{itemize}
\end{frame}

\begin{frame}
    \frametitle{杜老师}
    \begin{itemize}
        \item 根据上一道题我们很容易得出一个暴力:将$[L, R]$中的所有数直接插入线性基。复杂度爆炸。
        \item 观察到每个数大于$\sqrt R$的质因子只有至多一个,把这个质因子拿出来。
        \item 对于$i$来说,假设它有一个大于$\sqrt R$的质因子$p$,如果$i$是$[L, R]$中第一个$p$的指数为奇数的数,那么$i$的插入必然成功。我们称这样的位置为特殊位置。
        \item 否则,将$i$的状态异或上$p$对应的特殊位置的状态,将剩余的状态插入线性基。这样就可以只存$\sqrt R$个元素。
    \end{itemize}
\end{frame}

\begin{frame}
    \frametitle{杜老师}
    \begin{itemize}
        \item 我们不妨大胆猜想:存在一个阈值$k$,使得当$R - L > k$时,第$k$个数之后所有的非特殊位置插入必然失败,即矩阵已经满秩。
        \item 经验证,这个$k$在$6000$左右。
        \item 此时只需对于每个大于$\sqrt R$的$p$判断区间内是否有$p$的倍数。
    \end{itemize}
\end{frame}

\end{document}