\documentclass[12pt]{article}
\usepackage{xeCJK}
\usepackage{float}
\usepackage{indentfirst}
\usepackage{booktabs}
\usepackage{geometry}  
\usepackage{titlesec}   
\usepackage{array}
\usepackage{multirow}
\usepackage{listings}
\usepackage{titlesec} 
% \pagestyle{fancy} 
\geometry{left=2.5cm,right=3.5cm,top=2.5cm,bottom=2.5cm}

\title{{\huge 膜您赛}}
\author{{\large Two Pigeons}}
\setlength{\parindent}{2em}
\linespread{1.3}

\date{}
\begin{document}
\newpagestyle{main}{            
    \sethead{膜您赛}{}{\sectiontitle}     %设置页眉
    \setfoot{}{第\thepage 页,共8页}{}      %设置页脚,可以在页脚添加 \thepage  显示页数
    \headrule                                      % 添加页眉的下划线
    \footrule                                       %添加页脚的下划线
}
\pagestyle{main}    %使用该style

	\begin{titlepage}
		\maketitle
		\begin{table}[!htbp]
			\centering
			\begin{tabular}{|p{3.5cm}|p{3.5cm}|p{3.5cm}|p{3.5cm}|}
				\hline
				题目名称 & 树与路径 & 树据结构 & 树上的数\\
				\hline
				题目类型 & 传统型 & 传统型 & 传统型\\
				\hline
				输入文件名 & path.in & tree.in & number.in\\
				\hline
				输出文件名 & path.out & tree.out & number.out\\
				\hline
				源程序文件名 & path.cpp/c/pas & tree.cpp/c/pas & number.cpp/c/pas\\
				\hline
				每个测试点时限 & 1.0s & 1.0s & 8.0s \\
				\hline
				内存限制 & 256MB & 1G & 1.5G \\
				\hline
				子任务数目 & 7 & 8 & 9 \\
				\hline
			\end{tabular}
		\end{table}
		编译命令
		\begin{table}[!htbp]
			\centering
			\begin{tabular}{|p{3.5cm}|p{11.3cm}<{\centering}|} 
				\hline
				对于C++ & \textsf{g++ -O2 -std=c++11 -Wl,--stack=1024000000 *.cpp -o *}\\
				\hline
				对于其它语言 & \textsf{rm -rf *}\\
				\hline
			\end{tabular}
		\end{table}
	\end{titlepage}
	\section{T1 树与路径 (path)}
	给出一棵$n$个点的树,定义$f(k)$表示用恰好$k$条边不相交的路径覆盖整棵树的方案数\par
	注意每条路径的起点和终点不能重合,一棵树被覆盖当且仅当每条边都被覆盖\par
	对于每个$k\in[1, n)$,你需要求出$f(k)$的值\par
	答案对$998244353$取模
	\subsection{输入格式}
	从path.in中读入数据\par
	第一行包含一个整数$n$,表示树的节点数\par
	接下来$n - 1$行,每行两个整数$u, v$,表示树上的一条边\par
	\subsection{输出格式}
	输出到文件path.out中\par
	输出一行,包含$n - 1$个整数,第$i$个数表示$f(i)$对$998244353$取模之后的值
	\subsection{样例1 输入}
	\begin{lstlisting}
    6
    1 2
    2 3
    1 4
    2 5
    1 6		
	\end{lstlisting}
	\subsection{样例1 输出}
	\begin{lstlisting}
    0 0 9 6 1 	
	\end{lstlisting}
	\subsection{样例2 输入}
	\begin{lstlisting}
    12
    6 1
    6 7
    1 12
    6 3
    7 11
    12 5
    1 8
    12 2
    1 10
    1 4
    3 9	
	\end{lstlisting}
	\subsection{样例2 输出}
	\begin{lstlisting}
    0 0 0 0 135 450 579 364 117 18 1 	
	\end{lstlisting}
	\subsection{样例3}
	见下发文件中的path/path3.in以及path/path3.ans
	\subsection{子任务}
	对于所有数据,保证$n\leq 10^5$\par
	\begin{table}[H]
		\centering
		\begin{tabular}{cccc}
			\toprule
			子任务 & 分数  & $n$ & 特殊性质\\
			\midrule
			1 & 2 & $= 1$ & 无 \\
			\midrule
			2 & 8 & $\leq 7$ & 无 \\
			\midrule
			3 & 10 & $\leq 100$ & 无 \\
			\midrule
			4 & 15 & $\leq 1000$ & 无 \\
			\midrule
			5 & 10 & $10^5$ & 给出的树是一条链 \\
			\midrule
			6 & 25 & $5\times 10^4$ & 无 \\
			\midrule
			7 & 30 & $10^5$ & 无 \\
			\bottomrule
		\end{tabular}
	\end{table}\par
	\newpage
	\section{T2 树据结构 (tree)}
	有两棵树,我们称第一棵树为“模板树”,第二棵树为“大树”。两棵树一开始都各自有$n$个节点并且完全相同,我们认为$1$号节点是根节点\par
	你需要支持$4$种操作\par
	\begin{itemize}
		\item $1\ a\ b\ c$:修改模板树上的一条路径,即给$a$到$b$路径上的所有边的权值都加上$c$,保证$a\neq b$
		\item $2\ a\ w$:将当前的模板树复制一份(包括点权),然后接在大树中的节点$a$下方,边权为$w$。假设大树在接入之前共有$m$个节点,那么此次新接入的节点将按照原来的编号顺序重新分配$m + 1\sim m + n$的编号
		\item $3\ a\ b$:查询大树上$a$到$b$的路径的边权之和,保证$a\neq b$
		\item $4\ a$:查询大树上$a$的子树中的边权之和,如果$a$是叶子,那么答案为$0$
	\end{itemize}\par
	对于部分数据要求强制在线
	\subsection{输入格式}
	从tree.in中读入数据\par
	第一行三个整数$n, q, type$,分别表示节点数、询问个数以及强制在线参数\par
	接下来$n - 1$行,每行两个整数$u, v, w$,表示模板树和大树初始状态下的一条边\par
	接下来$q$行,每行表示一组询问,询问的格式见题目描述\par
	为了体现程序的在线性,每组询问除第一个数以外,其它的每个参数均会异或上$lastans\times type$。$lastans$表示上一次询问操作的答案,它的初值为$0$
	\subsection{输出格式}
	输出到文件tree.out中\par
	对于每个询问操作输出该询问的答案
	\subsection{样例1 输入}
	\begin{lstlisting}
    4 5 0
    1 2 1
    2 3 2
    2 4 3
    1 1 3 2
    2 4 5
    3 8 3
    4 2
    3 4 5
	\end{lstlisting}
	\subsection{样例1 输出}
	\begin{lstlisting}
    16
    20
    5
	\end{lstlisting}
	\subsection{样例2 输入}
	\begin{lstlisting}
    5 10 0
    1 2 6
    2 3 9
    1 4 8
    3 5 10
    1 2 1 3
    4 2
    2 4 2
    4 8
    2 8 3
    1 3 1 2
    3 3 2
    1 3 1 8
    4 2
    3 15 2	
	\end{lstlisting}
	\subsection{样例2 输出}
	\begin{lstlisting}
    19
    10
    9
    19
    65
	\end{lstlisting}
	\subsection{样例 3}
	见下发文件中的tree/tree3.in以及tree/tree3.ans\par
	\subsection{样例 4}
	见下发文件中的tree/tree4.in以及tree/tree4.ans\par
	\subsection{样例 5}
	见下发文件中的tree/tree5.in以及tree/tree5.ans\par
	\subsection{子任务}
	对于所有数据,保证$n, q\leq 10^5, w_i, c\leq 10^3$\par
	\begin{table}[H]
		\centering
		\begin{tabular}{cccc}
			\toprule
			子任务 & 分数  & $n, q$ & 特殊性质\\
			\midrule
			1 & 3 & $\leq 100$ & 无 \\
			\midrule
			2 & 7 & $\leq 2000$ & $type = 0$ \\
			\midrule
			3 & 8 & $n = 1, q\leq 10^5$ & 无 \\
			\midrule
			4 & 12 & $\leq 10^5$ & 没有询问操作$4$,且$type = 0$\\
			\midrule
			5 & 8 & $\leq 10^5$ & 没有询问操作$4$\\
			\midrule
			6 & 19 & $\leq 10^5$ & 没有询问操作$3$,且$type = 0$\\
			\midrule
			7 & 28 & $\leq 10^5$ & 没有询问操作$3$\\
			\midrule
			8 & 15 & $\leq 10^5$ & 无\\
			\bottomrule
		\end{tabular}
	\end{table}  
	\newpage
	\section{T3 树上的数 (number)}
	有一棵$n$个节点的树,你需要为每个节点指定一个权值,这个权值必须为\textbf{正奇数}且所有节点权值的乘积不能大于$m$\par
	对于一条路径来说,我们定义这条路径的权值为它经过的点的点权的$\gcd$\par
	定义一种分配权值的方案的答案为所有路径的权值的乘积。注意路径是无序的,这意味着$(u, v)$与$(v, u)$是同一条路径,但路径的起点和终点可以相同\par
	你需要求出所有合法的方案的权值之和,两种方案不同当且仅当存在至少一个点,这个点在两种方案中被分配了不同的权值\par
	答案对$998244353$取模\par
	\subsection{输入格式}
	从number.in中读入数据\par
	第一行两个整数$n, m$\par
	接下来$n - 1$行,每行两个整数$(u, v)$表示一条边\par
	\subsection{输出格式}
	输出到文件number.out中\par
	输出一个数,表示答案对$998244353$取模之后的值\par
	\subsection{样例1 输入}
	\begin{lstlisting}
    4 21
    1 2
    1 3
    1 4	
	\end{lstlisting}
	\subsection{样例1 输出}
	\begin{lstlisting}
    1021	
	\end{lstlisting}
	\subsection{样例2 输入}
	\begin{lstlisting}
    10 97
    1 2
    2 3
    4 6
    3 5
    2 7
    6 8
    3 6
    1 9
    2 10		
	\end{lstlisting}
	\subsection{样例2 输出}
	\begin{lstlisting}
    1691395	
	\end{lstlisting}
	\subsection{样例3 输入}
	\begin{lstlisting}
    10 23333333
    10 7
    10 8
    10 9
    8 6
    10 1
    8 5
    7 3
    7 4
    4 2	
	\end{lstlisting}
	\subsection{样例3 输出}
	\begin{lstlisting}
    120423471	
	\end{lstlisting}
	\subsection{样例4}
	见下发文件中的number/number4.in以及number/number4.ans\par
	\subsection{子任务}
	对于所有数据,保证$n\leq 100, m\leq 10^{10}$,保证树的形态随机\par
	\begin{table}[H]
		\centering
		\begin{tabular}{ccc}
			\toprule
			子任务 & 分数  & $n, m$ \\
			\midrule
			1 & 3 & $n = 1$ \\
			\midrule
			2 & 7 & $m\leq 100, n\leq 10$ \\
			\midrule
			3 & 10 & $m\leq 10^6, n\leq 10$ \\
			\midrule
			4 & 15 & $m\leq 10^{10}, n\leq 10$ \\
			\midrule
			5 & 15 & $m\leq 10^6, n\leq 50$ \\
			\midrule
			6 & 15 & $m\leq 10^9, n\leq 50$ \\
			\midrule
			7 & 15 & $m\leq 10^9$ \\
			\midrule
			8 & 10 & $n\leq 50$ \\
			\midrule
			9 & 10 & $m\leq 10^{10}, n\leq 100$ \\
			\bottomrule
		\end{tabular}
	\end{table}\par     
\end{document}